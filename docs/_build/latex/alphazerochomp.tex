%% Generated by Sphinx.
\def\sphinxdocclass{report}
\documentclass[letterpaper,10pt,english]{sphinxmanual}
\ifdefined\pdfpxdimen
   \let\sphinxpxdimen\pdfpxdimen\else\newdimen\sphinxpxdimen
\fi \sphinxpxdimen=.75bp\relax
\ifdefined\pdfimageresolution
    \pdfimageresolution= \numexpr \dimexpr1in\relax/\sphinxpxdimen\relax
\fi
%% let collapsible pdf bookmarks panel have high depth per default
\PassOptionsToPackage{bookmarksdepth=5}{hyperref}

\PassOptionsToPackage{booktabs}{sphinx}
\PassOptionsToPackage{colorrows}{sphinx}

\PassOptionsToPackage{warn}{textcomp}
\usepackage[utf8]{inputenc}
\ifdefined\DeclareUnicodeCharacter
% support both utf8 and utf8x syntaxes
  \ifdefined\DeclareUnicodeCharacterAsOptional
    \def\sphinxDUC#1{\DeclareUnicodeCharacter{"#1}}
  \else
    \let\sphinxDUC\DeclareUnicodeCharacter
  \fi
  \sphinxDUC{00A0}{\nobreakspace}
  \sphinxDUC{2500}{\sphinxunichar{2500}}
  \sphinxDUC{2502}{\sphinxunichar{2502}}
  \sphinxDUC{2514}{\sphinxunichar{2514}}
  \sphinxDUC{251C}{\sphinxunichar{251C}}
  \sphinxDUC{2572}{\textbackslash}
\fi
\usepackage{cmap}
\usepackage[T1]{fontenc}
\usepackage{amsmath,amssymb,amstext}
\usepackage{babel}



\usepackage{tgtermes}
\usepackage{tgheros}
\renewcommand{\ttdefault}{txtt}



\usepackage[Bjarne]{fncychap}
\usepackage{sphinx}

\fvset{fontsize=auto}
\usepackage{geometry}


% Include hyperref last.
\usepackage{hyperref}
% Fix anchor placement for figures with captions.
\usepackage{hypcap}% it must be loaded after hyperref.
% Set up styles of URL: it should be placed after hyperref.
\urlstyle{same}

\addto\captionsenglish{\renewcommand{\contentsname}{Contents:}}

\usepackage{sphinxmessages}
\setcounter{tocdepth}{1}



\title{AlphaZeroChomp}
\date{Jun 15, 2024}
\release{1.0.0}
\author{Alessandro Canzonieri}
\newcommand{\sphinxlogo}{\vbox{}}
\renewcommand{\releasename}{Release}
\makeindex
\begin{document}

\ifdefined\shorthandoff
  \ifnum\catcode`\=\string=\active\shorthandoff{=}\fi
  \ifnum\catcode`\"=\active\shorthandoff{"}\fi
\fi

\pagestyle{empty}
\sphinxmaketitle
\pagestyle{plain}
\sphinxtableofcontents
\pagestyle{normal}
\phantomsection\label{\detokenize{index::doc}}


\sphinxAtStartPar
Contents:

\sphinxstepscope


\chapter{AlphaZeroChomp}
\label{\detokenize{modules:module-main}}\label{\detokenize{modules:alphazerochomp}}\label{\detokenize{modules::doc}}\index{module@\spxentry{module}!main@\spxentry{main}}\index{main@\spxentry{main}!module@\spxentry{module}}\index{CustomArgumentParser (class in main)@\spxentry{CustomArgumentParser}\spxextra{class in main}}

\begin{fulllineitems}
\phantomsection\label{\detokenize{modules:main.CustomArgumentParser}}
\pysigstartsignatures
\pysiglinewithargsret{\sphinxbfcode{\sphinxupquote{class\DUrole{w}{ }}}\sphinxcode{\sphinxupquote{main.}}\sphinxbfcode{\sphinxupquote{CustomArgumentParser}}}{\sphinxparam{\DUrole{n}{prog=None}}\sphinxparamcomma \sphinxparam{\DUrole{n}{usage=None}}\sphinxparamcomma \sphinxparam{\DUrole{n}{description=None}}\sphinxparamcomma \sphinxparam{\DUrole{n}{epilog=None}}\sphinxparamcomma \sphinxparam{\DUrole{n}{parents={[}{]}}}\sphinxparamcomma \sphinxparam{\DUrole{n}{formatter\_class=\textless{}class \textquotesingle{}argparse.HelpFormatter\textquotesingle{}\textgreater{}}}\sphinxparamcomma \sphinxparam{\DUrole{n}{prefix\_chars=\textquotesingle{}\sphinxhyphen{}\textquotesingle{}}}\sphinxparamcomma \sphinxparam{\DUrole{n}{fromfile\_prefix\_chars=None}}\sphinxparamcomma \sphinxparam{\DUrole{n}{argument\_default=None}}\sphinxparamcomma \sphinxparam{\DUrole{n}{conflict\_handler=\textquotesingle{}error\textquotesingle{}}}\sphinxparamcomma \sphinxparam{\DUrole{n}{add\_help=True}}\sphinxparamcomma \sphinxparam{\DUrole{n}{allow\_abbrev=True}}\sphinxparamcomma \sphinxparam{\DUrole{n}{exit\_on\_error=True}}}{}
\pysigstopsignatures\index{error() (main.CustomArgumentParser method)@\spxentry{error()}\spxextra{main.CustomArgumentParser method}}

\begin{fulllineitems}
\phantomsection\label{\detokenize{modules:main.CustomArgumentParser.error}}
\pysigstartsignatures
\pysiglinewithargsret{\sphinxbfcode{\sphinxupquote{error}}}{\sphinxparam{\DUrole{n}{message}\DUrole{p}{:}\DUrole{w}{ }\DUrole{n}{string}}}{}
\pysigstopsignatures
\sphinxAtStartPar
Prints a usage message incorporating the message to stderr and
exits.

\sphinxAtStartPar
If you override this in a subclass, it should not return \textendash{} it
should either exit or raise an exception.

\end{fulllineitems}


\end{fulllineitems}

\index{module@\spxentry{module}!AlphaZero@\spxentry{AlphaZero}}\index{AlphaZero@\spxentry{AlphaZero}!module@\spxentry{module}}\index{module@\spxentry{module}!Alpha\_GraphMCTS@\spxentry{Alpha\_GraphMCTS}}\index{Alpha\_GraphMCTS@\spxentry{Alpha\_GraphMCTS}!module@\spxentry{module}}\index{module@\spxentry{module}!Resnet@\spxentry{Resnet}}\index{Resnet@\spxentry{Resnet}!module@\spxentry{module}}\index{ResBlock (class in Resnet)@\spxentry{ResBlock}\spxextra{class in Resnet}}\phantomsection\label{\detokenize{modules:module-AlphaZero}}\phantomsection\label{\detokenize{modules:module-Alpha_GraphMCTS}}\phantomsection\label{\detokenize{modules:module-Resnet}}

\begin{fulllineitems}
\phantomsection\label{\detokenize{modules:Resnet.ResBlock}}
\pysigstartsignatures
\pysiglinewithargsret{\sphinxbfcode{\sphinxupquote{class\DUrole{w}{ }}}\sphinxcode{\sphinxupquote{Resnet.}}\sphinxbfcode{\sphinxupquote{ResBlock}}}{\sphinxparam{\DUrole{n}{num\_hidden}}}{}
\pysigstopsignatures
\sphinxAtStartPar
A Residual Block for a neural network.
\begin{quote}\begin{description}
\sphinxlineitem{Parameters}
\sphinxAtStartPar
\sphinxstyleliteralstrong{\sphinxupquote{num\_hidden}} (\sphinxstyleliteralemphasis{\sphinxupquote{int}}) \textendash{} Number of hidden units in the convolutional layers.

\end{description}\end{quote}
\index{forward() (Resnet.ResBlock method)@\spxentry{forward()}\spxextra{Resnet.ResBlock method}}

\begin{fulllineitems}
\phantomsection\label{\detokenize{modules:Resnet.ResBlock.forward}}
\pysigstartsignatures
\pysiglinewithargsret{\sphinxbfcode{\sphinxupquote{forward}}}{\sphinxparam{\DUrole{n}{x}}}{}
\pysigstopsignatures
\sphinxAtStartPar
Performs the forward pass of the residual block.

\end{fulllineitems}

\index{forward() (Resnet.ResBlock method)@\spxentry{forward()}\spxextra{Resnet.ResBlock method}}

\begin{fulllineitems}
\phantomsection\label{\detokenize{modules:id0}}
\pysigstartsignatures
\pysiglinewithargsret{\sphinxbfcode{\sphinxupquote{forward}}}{\sphinxparam{\DUrole{n}{x}}}{}
\pysigstopsignatures
\sphinxAtStartPar
The \sphinxtitleref{forward} function defines the forward pass of the ResNet block.

\end{fulllineitems}


\end{fulllineitems}

\index{ResNet (class in Resnet)@\spxentry{ResNet}\spxextra{class in Resnet}}

\begin{fulllineitems}
\phantomsection\label{\detokenize{modules:Resnet.ResNet}}
\pysigstartsignatures
\pysiglinewithargsret{\sphinxbfcode{\sphinxupquote{class\DUrole{w}{ }}}\sphinxcode{\sphinxupquote{Resnet.}}\sphinxbfcode{\sphinxupquote{ResNet}}}{\sphinxparam{\DUrole{n}{args}}}{}
\pysigstopsignatures\index{create\_coordinate\_matrix() (Resnet.ResNet method)@\spxentry{create\_coordinate\_matrix()}\spxextra{Resnet.ResNet method}}

\begin{fulllineitems}
\phantomsection\label{\detokenize{modules:Resnet.ResNet.create_coordinate_matrix}}
\pysigstartsignatures
\pysiglinewithargsret{\sphinxbfcode{\sphinxupquote{create\_coordinate\_matrix}}}{}{}
\pysigstopsignatures
\sphinxAtStartPar
Creates a symmetric matrix with unique diagonal elements and symmetric off\sphinxhyphen{}diagonal elements.

\sphinxAtStartPar
The matrix is of size \sphinxtitleref{(max\_size, max\_size)} with diagonal elements from \sphinxtitleref{0} to \sphinxtitleref{max\_size \sphinxhyphen{} 1}
and symmetric off\sphinxhyphen{}diagonal elements that encode positional symmetries.
\begin{quote}\begin{description}
\sphinxlineitem{Returns}
\sphinxAtStartPar
A symmetric matrix with unique diagonal elements and symmetric off\sphinxhyphen{}diagonal elements.

\sphinxlineitem{Return type}
\sphinxAtStartPar
torch.Tensor

\end{description}\end{quote}

\end{fulllineitems}

\index{forward() (Resnet.ResNet method)@\spxentry{forward()}\spxextra{Resnet.ResNet method}}

\begin{fulllineitems}
\phantomsection\label{\detokenize{modules:Resnet.ResNet.forward}}
\pysigstartsignatures
\pysiglinewithargsret{\sphinxbfcode{\sphinxupquote{forward}}}{\sphinxparam{\DUrole{n}{x\_input}}\sphinxparamcomma \sphinxparam{\DUrole{n}{current\_actions}\DUrole{o}{=}\DUrole{default_value}{None}}\sphinxparamcomma \sphinxparam{\DUrole{n}{opponent\_actions}\DUrole{o}{=}\DUrole{default_value}{None}}}{}
\pysigstopsignatures
\sphinxAtStartPar
Define the computation performed at every call.

\sphinxAtStartPar
Should be overridden by all subclasses.

\begin{sphinxadmonition}{note}{Note:}
\sphinxAtStartPar
Although the recipe for forward pass needs to be defined within
this function, one should call the \sphinxcode{\sphinxupquote{Module}} instance afterwards
instead of this since the former takes care of running the
registered hooks while the latter silently ignores them.
\end{sphinxadmonition}

\end{fulllineitems}

\index{mask\_and\_renorm\_NoSoftmax() (Resnet.ResNet method)@\spxentry{mask\_and\_renorm\_NoSoftmax()}\spxextra{Resnet.ResNet method}}

\begin{fulllineitems}
\phantomsection\label{\detokenize{modules:Resnet.ResNet.mask_and_renorm_NoSoftmax}}
\pysigstartsignatures
\pysiglinewithargsret{\sphinxbfcode{\sphinxupquote{mask\_and\_renorm\_NoSoftmax}}}{\sphinxparam{\DUrole{n}{x\_input}}\sphinxparamcomma \sphinxparam{\DUrole{n}{unrenorm\_policy}}}{}
\pysigstopsignatures
\sphinxAtStartPar
Masks and renormalizes the policy without using the Softmax function.

\sphinxAtStartPar
This method applies a mask to the policy to exclude invalid moves and renormalizes
the policy without using the Softmax function. It works best with args{[}‘MCTS\_set\_equal\_prior’{]} = True
and allows faster convergence of the ResNet network. This function can handle both batched and unbatched inputs.
\begin{quote}\begin{description}
\sphinxlineitem{Parameters}\begin{itemize}
\item {} 
\sphinxAtStartPar
\sphinxstyleliteralstrong{\sphinxupquote{x\_input}} (\sphinxstyleliteralemphasis{\sphinxupquote{torch.Tensor}}) \textendash{} Input tensor representing the state, either batched (3D) or unbatched (2D).

\item {} 
\sphinxAtStartPar
\sphinxstyleliteralstrong{\sphinxupquote{unrenorm\_policy}} (\sphinxstyleliteralemphasis{\sphinxupquote{torch.Tensor}}) \textendash{} Unrenormalized policy tensor to be masked and renormalized.

\end{itemize}

\sphinxlineitem{Returns}
\sphinxAtStartPar
Masked and renormalized policy tensor.

\sphinxlineitem{Return type}
\sphinxAtStartPar
torch.Tensor

\end{description}\end{quote}

\end{fulllineitems}

\index{mask\_and\_renormalize() (Resnet.ResNet method)@\spxentry{mask\_and\_renormalize()}\spxextra{Resnet.ResNet method}}

\begin{fulllineitems}
\phantomsection\label{\detokenize{modules:Resnet.ResNet.mask_and_renormalize}}
\pysigstartsignatures
\pysiglinewithargsret{\sphinxbfcode{\sphinxupquote{mask\_and\_renormalize}}}{\sphinxparam{\DUrole{n}{x\_input}}\sphinxparamcomma \sphinxparam{\DUrole{n}{unrenorm\_policy}}}{}
\pysigstopsignatures
\sphinxAtStartPar
Masks and renormalizes the policy using the Softmax function.

\sphinxAtStartPar
This method applies a mask to the policy to exclude invalid moves, then renormalizes
the policy using the Softmax function. It can handle both batched and unbatched inputs.
\begin{quote}\begin{description}
\sphinxlineitem{Parameters}\begin{itemize}
\item {} 
\sphinxAtStartPar
\sphinxstyleliteralstrong{\sphinxupquote{x\_input}} (\sphinxstyleliteralemphasis{\sphinxupquote{torch.Tensor}}) \textendash{} Input tensor representing the state, either batched (3D) or unbatched (2D).

\item {} 
\sphinxAtStartPar
\sphinxstyleliteralstrong{\sphinxupquote{unrenorm\_policy}} (\sphinxstyleliteralemphasis{\sphinxupquote{torch.Tensor}}) \textendash{} Unrenormalized policy tensor to be masked and renormalized.

\end{itemize}

\sphinxlineitem{Returns}
\sphinxAtStartPar
Masked and renormalized policy tensor.

\sphinxlineitem{Return type}
\sphinxAtStartPar
torch.Tensor

\end{description}\end{quote}

\end{fulllineitems}

\index{prepare\_data() (Resnet.ResNet method)@\spxentry{prepare\_data()}\spxextra{Resnet.ResNet method}}

\begin{fulllineitems}
\phantomsection\label{\detokenize{modules:Resnet.ResNet.prepare_data}}
\pysigstartsignatures
\pysiglinewithargsret{\sphinxbfcode{\sphinxupquote{prepare\_data}}}{\sphinxparam{\DUrole{n}{x\_input}}\sphinxparamcomma \sphinxparam{\DUrole{n}{current\_actions}\DUrole{o}{=}\DUrole{default_value}{None}}\sphinxparamcomma \sphinxparam{\DUrole{n}{opponent\_actions}\DUrole{o}{=}\DUrole{default_value}{None}}}{}
\pysigstopsignatures
\sphinxAtStartPar
Prepares input data for a neural network model by incorporating current and opponent actions.
\begin{quote}\begin{description}
\sphinxlineitem{Parameters}\begin{itemize}
\item {} 
\sphinxAtStartPar
\sphinxstyleliteralstrong{\sphinxupquote{x\_input}} (\sphinxstyleliteralemphasis{\sphinxupquote{torch.Tensor}}) \textendash{} The state of the game.

\item {} 
\sphinxAtStartPar
\sphinxstyleliteralstrong{\sphinxupquote{current\_actions}} (\sphinxstyleliteralemphasis{\sphinxupquote{list}}\sphinxstyleliteralemphasis{\sphinxupquote{ or }}\sphinxstyleliteralemphasis{\sphinxupquote{torch.Tensor}}\sphinxstyleliteralemphasis{\sphinxupquote{, }}\sphinxstyleliteralemphasis{\sphinxupquote{optional}}) \textendash{} Actions taken by the current player. For self\sphinxhyphen{}play, it is a list of tuples with coordinates of actions.
For training, it is a matrix of dimensions (max\_size, max\_size) with +1 at action coordinates.

\item {} 
\sphinxAtStartPar
\sphinxstyleliteralstrong{\sphinxupquote{opponent\_actions}} (\sphinxstyleliteralemphasis{\sphinxupquote{list}}\sphinxstyleliteralemphasis{\sphinxupquote{ or }}\sphinxstyleliteralemphasis{\sphinxupquote{torch.Tensor}}\sphinxstyleliteralemphasis{\sphinxupquote{, }}\sphinxstyleliteralemphasis{\sphinxupquote{optional}}) \textendash{} Actions taken by the opponent player. For self\sphinxhyphen{}play, it is a list of tuples with coordinates of actions.
For training, it is a matrix of dimensions (max\_size, max\_size) with \sphinxhyphen{}1 at action coordinates.

\end{itemize}

\sphinxlineitem{Returns}
\sphinxAtStartPar
The modified input data after processing, which includes the current player’s actions, opponent’s actions,
the original input state, and a coordinates matrix stacked together.

\sphinxlineitem{Return type}
\sphinxAtStartPar
torch.Tensor

\end{description}\end{quote}

\end{fulllineitems}

\index{set\_device() (Resnet.ResNet method)@\spxentry{set\_device()}\spxextra{Resnet.ResNet method}}

\begin{fulllineitems}
\phantomsection\label{\detokenize{modules:Resnet.ResNet.set_device}}
\pysigstartsignatures
\pysiglinewithargsret{\sphinxbfcode{\sphinxupquote{set\_device}}}{}{}
\pysigstopsignatures
\sphinxAtStartPar
Set the device for the PyTorch model based on the specified model device and availability.

\sphinxAtStartPar
This function updates the model to use ‘mps’ if specified and available, otherwise ‘cuda’
if specified and available, and defaults to ‘cpu’ if neither is available.
\begin{quote}\begin{description}
\sphinxlineitem{Parameters}
\sphinxAtStartPar
\sphinxstyleliteralstrong{\sphinxupquote{None}}

\sphinxlineitem{Return type}
\sphinxAtStartPar
None

\end{description}\end{quote}

\end{fulllineitems}

\index{set\_seed() (Resnet.ResNet method)@\spxentry{set\_seed()}\spxextra{Resnet.ResNet method}}

\begin{fulllineitems}
\phantomsection\label{\detokenize{modules:Resnet.ResNet.set_seed}}
\pysigstartsignatures
\pysiglinewithargsret{\sphinxbfcode{\sphinxupquote{set\_seed}}}{}{}
\pysigstopsignatures
\sphinxAtStartPar
Set all possible seeds to make the experiment reproducibol

\end{fulllineitems}


\end{fulllineitems}

\index{module@\spxentry{module}!Alpha\_Chomp\_Env@\spxentry{Alpha\_Chomp\_Env}}\index{Alpha\_Chomp\_Env@\spxentry{Alpha\_Chomp\_Env}!module@\spxentry{module}}\phantomsection\label{\detokenize{modules:module-Alpha_Chomp_Env}}

\chapter{Indices and tables}
\label{\detokenize{index:indices-and-tables}}\begin{itemize}
\item {} 
\sphinxAtStartPar
\DUrole{xref,std,std-ref}{genindex}

\item {} 
\sphinxAtStartPar
\DUrole{xref,std,std-ref}{modindex}

\item {} 
\sphinxAtStartPar
\DUrole{xref,std,std-ref}{search}

\end{itemize}


\renewcommand{\indexname}{Python Module Index}
\begin{sphinxtheindex}
\let\bigletter\sphinxstyleindexlettergroup
\bigletter{a}
\item\relax\sphinxstyleindexentry{Alpha\_Chomp\_Env}\sphinxstyleindexpageref{modules:\detokenize{module-Alpha_Chomp_Env}}
\item\relax\sphinxstyleindexentry{Alpha\_GraphMCTS}\sphinxstyleindexpageref{modules:\detokenize{module-Alpha_GraphMCTS}}
\item\relax\sphinxstyleindexentry{AlphaZero}\sphinxstyleindexpageref{modules:\detokenize{module-AlphaZero}}
\indexspace
\bigletter{m}
\item\relax\sphinxstyleindexentry{main}\sphinxstyleindexpageref{modules:\detokenize{module-main}}
\indexspace
\bigletter{r}
\item\relax\sphinxstyleindexentry{Resnet}\sphinxstyleindexpageref{modules:\detokenize{module-Resnet}}
\end{sphinxtheindex}

\renewcommand{\indexname}{Index}
\printindex
\end{document}